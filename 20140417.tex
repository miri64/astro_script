\subsection[Horizontsystem]{Horizontsystem (lokales irdisches Koordinatensystem)}
\begin{description}
    \item[Zenitvektor] $\vec n_z$, Richtung: Erdmittelpunkt -- Beobartungsposition
         Erdoberfläche
     \item[Nadir] Fußpunkt
     \item[Horizontebene]
         \begin{itemize}
             \item Vektor in Nordrichung: $\vec n_N$
             \item Vektor in Westrichtung: $\vec n_W = \vec n_z \times \vec n_N$
         \end{itemize}
     \item[Azimut] $\varphi$
     \item[Höhe] $h$
     \item[Zenitdistanz] $z = 90\degree - h \Leftrightarrow z + h = 90\degree$
\end{description}

\begin{definition}
    Ortonormalsystem Horizontsystem: $(\vec n_z, \vec n_N, \vec n_W)$
\end{definition}


\begin{center}
    \begin{tikzpicture}[scale=4,tdplot_main_coords]
        \pgfmathsetmacro{\rvec}{0.8}
        \pgfmathsetmacro{\phivec}{120}
        \pgfmathsetmacro{\hvec}{30}

        \coordinate (O) at (0,0,0);
        \tdplotsetcoord{S}{\rvec}{\hvec}{\phivec}

        \draw [->, thick] (O) -- (1, 0, 0) node[anchor=north east] {$\vec n_N$};
        \draw [->, thick] (O) -- (0, 1, 0) node[anchor=north west] {$\vec n_W$};
        \draw [->, thick] (O) -- (0, 0, 1) node[anchor=south] {$\vec n_z$};

        \tdplotsetrotatedcoords{60}{40}{30}

        \tdplotdrawarc[dashed]{(O)}{1.5}{0}{360}{anchor=south west}{Horizontebene}
        \draw [->] (O) -- (S) node [pos=0.5, anchor=south east] {$r$};
        \draw [dotted] (O) -- (Sxy);
        \draw [dotted] (Sxy) -- (S) node [anchor=south west] {$\vec n$};
        \tdplotdrawarc[dotted]{(O)}{0.3}{0}{\phivec}{anchor=south east}{$\varphi$}
        \tdplotsetthetaplanecoords{\phivec}
        \tdplotdrawarc[dotted, tdplot_rotated_coords]{(O)}{0.2}{\hvec}{90}{anchor=30}{$h$}
        \tdplotdrawarc[<-, dotted, tdplot_rotated_coords]{(O)}{0.2}{0}{\hvec}{anchor=70}{$z$}
    \end{tikzpicture}

    \textbf{Position}: $(\varphi, z)$ oder $(\varphi, h)$
    \[ \vec n = \sin \delta \cdot \vec n_W + \cos \delta \cdot \cos \alpha \cdot \vec n_N + \cos \delta \cdot \sin \alpha \cdot \vec n_z \]
\end{center}

\begin{description}
    \item[wahrer Horizont] Ebene durch Erdmittelpunkt
    \item[schwere Nachteile]
        \begin{itemize}
            \item Koordinaten sind ortsabhängig
            \item Koordinaten sind zeitabhängig
        \end{itemize}
\end{description}

\subsection[Globale irdische Koordinaten]{Globale irdische Koordinaten $(l, b)$}
\begin{goal}
    Position des Beobachters $\vec n_z$ global angeben
\end{goal}
\begin{center}
    \begin{tikzpicture}[scale=4,tdplot_main_coords]
        \pgfmathsetmacro{\rvec}{0.8}
        \pgfmathsetmacro{\lvec}{100}
        \pgfmathsetmacro{\bvec}{16}

        \coordinate (O) at (0,0,0);
        \tdplotsetcoord{S}{\rvec}{\bvec}{\lvec}
        \tdplotsetcoord{G}{1.5}{51.5}{0}

        \draw [->, thick] (O) -- (1, 0, 0) node[anchor=north east] {$\vec n_1$};
        \draw [->, thick] (O) -- (0, 1, 0) node[anchor=north west] {$\vec n_2$};
        \draw [->, thick] (O) -- (0, 0, 1) node[anchor=south] {$\vec n_e$};

        \tdplotsetrotatedcoords{60}{40}{30}

        \tdplotdrawarc[dashed]{(O)}{1.5}{0}{360}{anchor=south west}{Äquator}
        \draw [->] (O) -- (S) node [pos=0.5, anchor=south east] {$r$};
        \draw [->] (O) -- (G) node [anchor=south east] {$\vec n_z^G$};
        \draw [dotted] (O) -- (Sxy);
        \draw [dotted] (Sxy) -- (S) node [anchor=south west] {$\vec n_z$};
        \tdplotdrawarc[dotted]{(O)}{0.21}{0}{\lvec}{anchor=south east}{$l$}
        \tdplotsetthetaplanecoords{\lvec}
        \tdplotdrawarc[dotted, tdplot_rotated_coords]{(O)}{0.2}{\bvec}{90}{anchor=30}{$b$}
        \tdplotsetthetaplanecoords{0}
        \tdplotdrawarc[dashed, tdplot_rotated_coords]{(O)}{1.5}{90}{-90}{anchor=east}{Nullmeridian}
    \end{tikzpicture}
    \[ \vec n_z = \sin \delta \cdot \vec n_e + \cos \delta \cdot \cos \alpha \cdot \vec n_1 + \cos \delta \cdot \sin \alpha \cdot \vec n_2 \]
\end{center}
$\hookrightarrow$ 1. Nachteil: Ausgleich der Ortsabhängigkeit

\subsection{Beziehung zwischen den Koordinatensystemen}
\subsubsection{Projektion auf Himmelsäquator}
\begin{description}
    \item[Winkel $\Theta$] von Projektion von $\vec n_z$ bei $\vec n_{\vernal}$ \\
        $\rightarrow$ Sternzeit $[ \mathrm{h} : \mathrm{m} : \second ]$
    \item[Speziell] $\Theta_G$ Sternzeit Greenwich
        \[ \underbrace{\Theta(l)}_{\vec n_z} = \underbrace{\Theta_G}_{\vec n_z^G} - l \]
    \item[Sternzeit $\rightarrow$ Uhrzeit] Beispiel: $\Theta_G$
        \begin{align*}
            \Theta_G(t) &= \underbrace{\Theta_G(0)}_{t = 0\,\mathrm{h}} + t\left(\frac{367.24}{365.24}\right) \\
            1\,\mathrm{a} &= \underbrace{366.24\,\mathrm{d}}_{\text{Sterntage}} = \underbrace{365.24\,\mathrm{d}}_{\text{normale Tage}} \\
                          &\Rightarrow \text{1 Sterntag ist kürzer als ein normaler Tag}
        \end{align*}
\end{description}

\begin{itemize}
    \item $\tau$: Winkel zwischen projezierten $\vec n_z$ und projezierten $\vec n$ (Objekt) \\
        $\rightarrow$ Stundenwinkel\\
        $\Rightarrow$ Sternzeit ist der Stundenwinkel des Frühlingspunkts
    \item Beziehung zwischen den Koordinatensystemen:
        \[ \boxed{\tau = \Theta(l, t) - \alpha = \theta_G(0) + t \left(\frac{366.24}{365.24}\right) - l - \alpha} \]
        $\rightarrow$ Beobachtung: $(\tau, \delta)$
\end{itemize}

\subsection{Ekliptikales System}
\begin{description}
    \item[ekliptikale Länge] $\lambda$
    \item[ekliptikale Breite] $\beta$
\end{description}
\[ \varepsilon = 23\degree 27\arcsecond \]

\begin{center}
    \begin{tikzpicture}[scale=4,tdplot_main_coords]
        \pgfmathsetmacro{\rvec}{0.8}
        \pgfmathsetmacro{\lvec}{80}
        \pgfmathsetmacro{\bvec}{20}

        \coordinate (O) at (0,0,0);
        \tdplotsetcoord{S}{\rvec}{\bvec}{\lvec}
        \tdplotsetcoord{n_e}{1.5}{23.4}{0}

        \draw [->, thick] (O) -- (1, 0, 0) node[anchor=north east] {$\vec n_{\vernal}$};
        \draw [->, thick] (O) -- (0, 1, 0) node[anchor=north west] {$(\vec n_b \times \vec n_{\vernal})$};
        \draw [->, thick] (O) -- (0, 0, 1) node[anchor=south] {$\vec n_b$};

        \tdplotsetrotatedcoords{60}{40}{30}

        \tdplotdrawarc[dashed]{(O)}{1.5}{0}{360}{anchor=south west}{Äquator}
        \draw [->] (O) -- (S);
        \draw [->] (O) -- (n_e) node [anchor=south east] {$\vec n_e$};
        \draw [dotted] (O) -- (Sxy);
        \draw [dotted] (Sxy) -- (S);
        \tdplotdrawarc[dotted]{(O)}{0.21}{0}{\lvec}{anchor=south east}{$\lambda$}
        \tdplotsetthetaplanecoords{\lvec}
        \tdplotdrawarc[dotted, tdplot_rotated_coords]{(O)}{0.2}{\bvec}{90}{anchor=30}{$\beta$}
        \tdplotsetthetaplanecoords{0}
        \tdplotdrawarc[dashed, tdplot_rotated_coords]{(O)}{0.5}{0}{23.4}{anchor=north}{$\varepsilon$}
    \end{tikzpicture}
    \[ \vec n_z = \sin \delta \cdot \vec n_e + \cos \delta \cdot \cos \alpha \cdot \vec n_1 + \cos \delta \cdot \sin \alpha \cdot \vec n_2 \]
\end{center}

\subsection{Galaktische Koordinaten}
\begin{center}
    \begin{tikzpicture}[scale=4,tdplot_main_coords]
        \pgfmathsetmacro{\rvec}{0.8}
        \pgfmathsetmacro{\lvec}{115}
        \pgfmathsetmacro{\bvec}{70}

        \coordinate (O) at (0,0,0);
        \coordinate (GZ) at (0,1,0);
        \tdplotsetcoord{S}{\rvec}{\bvec}{\lvec}
        \tdplotsetcoord{n_e}{1.5}{23.4}{0}

        \tdplotsetrotatedcoords{60}{40}{30}

        \tdplotdrawarc{(GZ)}{1.5}{0}{360}{anchor=south west}{}
        \node at (O) {$\astrosun$};
        \draw [thick] (O) -- (GZ);
        \node at (GZ) {$\circ$};
        \node [anchor=west] at (GZ) {$GZ$};
        \node [anchor=north] at (GZ) {Sagitarius A$^*$};
        \draw [->] (O) -- (S) node {$\star$};
        \draw [->] (GZ) -- (0,1,1);
        \draw [dotted] (O) -- (Sxy);
        \draw [dotted] (Sxy) -- (S);
        \tdplotdrawarc[dotted]{(O)}{0.7}{90}{\lvec}{anchor=east}{$l$}
        \tdplotsetthetaplanecoords{\lvec}
        \tdplotdrawarc[dotted, tdplot_rotated_coords]{(O)}{0.6}{\bvec}{90}{anchor=30}{$b$}
    \end{tikzpicture}
\end{center}

\paragraph{Galaktisches Zentrum (GZ)}
\begin{align*}
    \alpha &= 17^\mathrm{h}42^\mathrm{m}25^\second \\
    \delta &= -28\degree 55\arcminute
\end{align*}

\section{Bahnen der Himmelskörper im Horizontsystem}
\begin{center}
    \framebox{Erddrehung $\Rightarrow$ tägliche Bewegung}
\end{center}
\paragraph{Ausgezeichnete Punkte}
\begin{itemize}
    \item Aufgang und Untergang:
        \begin{itemize}
            \item Parallelkreis-Horizont
            \item Schnittpunkt
        \end{itemize}
    \item obere Kulmination: $h_1 = 90\degree - |b - \delta|$
    \item untere Kulmination: $h_1 = -90\degree + |b + \delta|$
\end{itemize}

\paragraph{Speziell}
\begin{itemize}
    \item Zirkumpolare Sterne: $\delta > 90\degree - |b|$
    \item nie sichtbar: $\delta < -90\degree + |b|$
\end{itemize}

\paragraph{Änderung der Höhe mit der Zeit}
\[ h(t) = \arcsin(\sin b \sin \delta + \cos b \cos \delta \cos \tau) \]

\paragraph{Näherung} ($h$ klein)
\[ h \approx \delta + (90\degree - b) \cos \tau \]
