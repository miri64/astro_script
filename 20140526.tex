\subsection{Direkte Beobachtung}
\begin{goal}
    "`Foto"' von der 2. Erde in der habitablen Zone (HZ)
\end{goal}

\paragraph{Problem} Stern $\leftrightarrow$ Planeten: Kontrastunterschied
(Stern überstrahlt den Planeten)

% TODO Skizze

\textsc{Wien}'sches Verschiebungsgesetz:
\[ T \cdot \lambda_{\max} = 0.29\,\milli\kelvin = \textbf{const} \]

\begin{itemize}
    \item IR -- bessere Unterscheidung
    \item weite Systeme: Räumliche Seperation Stern -- Planet
\end{itemize}

\paragraph{Problem} Planet im Allgemeinen kühl und leuchtschwach\\
$\hookrightarrow$ junge Planetensysteme (wegen $f_Q$)

\paragraph{Beispiele}
\begin{itemize}
    \item 2M1207\\
        Planetenkanditat um braunen Zwerg
    \item HR8799: Sonnensystem in massereicher "`Ausführung"'
        \begin{align*}
            b &: 10\,M_{\jupiter} \\
            c &: 10\,M_{\jupiter} \\
            d &: 7\,M_{\jupiter} 
        \end{align*}
\end{itemize}

\chapter[Doppelsterne]{Doppelsterne (DS)}
\section{Klassen von DS}
\begin{itemize}
    \item visuelle DS (direkte Beobarchtung)
    \item astronometrische DS
    \item spektroskopische DS
        \begin{itemize}
            \item Single line
            \item double line
        \end{itemize}
    \item Bedeckungsveränderlichkeit (Transit)
\end{itemize}

\section{Roche-Lobe-Massentransfer}
% TODO Skizze

\paragraph{Annahme} $M_{*,1} = M_{*,2}$; Gas: H (1-atomig)
\begin{align*}
    R &: \text{Radius-$RL$} \\
    y &= R - d \\
    A &= \pi x^2
\end{align*}

\paragraph{Annahme} $\boxed{d \ll R}$
\begin{align*}
    x^2 + \left(y + \frac{d}{2}\right)^2 &= R^2 \\
    x^2 + \left(R - d + \frac{d}{2}\right)^2 &= x^2 \left(R - \frac{d}{2}\right)^2 = R^2 \\
    \Rightarrow x^2 + \cancel{R^2} - \left(\cancel{R} - \frac{d}{2}\right)^2 &= R \cdot d - \underset{\approx 0}{\frac{d^2}{4}} \approx R \cdot d \\
    \Rightarrow x &= \sqrt{R \cdot d}
\end{align*}

\paragraph{Massentransfer}
\begin{align*}
    \dot M &= A \cdot \rho \cdot v \\
           &= \pi \cdot R \cdot d \cdot \rho \cdot v
\end{align*}

$v = v_{\mathrm{th}}$:
\begin{align*}
    \frac{m_{\mathrm{H}}}{\cancel{2}} v_{\mathrm{th}}^2 &= \frac{3}{\cancel{2}} kT \\
    v_{\mathrm{th}} &= \sqrt{\frac{3kT}{m_{\mathrm{H}}}} \\
    \Rightarrow \dot M &= \pi \underbrace{\underset{\uparrow}{\rho} R d \sqrt{\frac{3k\overset{\downarrow}{T}}{m_{\mathrm{H}}}}}_{\text{von $d$ abhängig}} \sim \boxed{d} \\
                       &\Rightarrow \boxed{\dot M \sim d^3}
\end{align*}

\part{Astronomische Beobachtungsgrößen}
\chapter{Astronomische Spektroskopie}
\section[Astronomische Helligskeitsysteme]{Astronomische Helligskeitsysteme (Bsp. Sterne)}
am Beispiel Sterne

% TODO Skizze

\[ L_* = F_*(R_*) \cdot 4 \pi R_*^2 = f(r) \cdot 4 \pi r^2 = \mathbf{const} \]

\[ \boxed{f(r) = F_*(R_*) \cdot \left(\frac{R_*}{r}\right)^2} \]

Strahlungsfluss -- Beobarchtungsgröße $\rightarrow$ \emph{nicht} Intensität

\paragraph{Ursache} keine räumliche Auflösung
\[ \Delta m = 1\magnitude \]

\framebox{Helligkeiten}: Größenklassen (Magnituden)

% TODO Skala

\begin{center}
    Astronomie $\longleftrightarrow$ Physik
\end{center}

\fbox{ %
    \begin{minipage}[h]{\textwidth} %
        \begin{align*}
            \Delta m = m_2 - m_1 &= -2.5\magnitude \log f_2 - \log f_1 \\
                                 &= -2.5\magnitude \log\left(\frac{f_2}{f_1}\right)
        \end{align*}
    \end{minipage}
}

$\Rightarrow m = -2.5\magnitude \log f + C$

Scheinbare Helligkeit: $m(r)$

Absolute Helligkeit: $M := m(10\,\parsec) \leftarrow$ Referenzentfernung

\subsubsection{Bolometrische Helligkeit}
\[ m_{\mathrm{bol}}(r) = -2.5 \log \int\limits_0^{\infty} f_\nu(r)\ d\nu + C \]

\subsubsection{Verschiedene Filterbereiche}
\[ m_x(r) = -2.5 \log \int\limits_{0}^{\infty} f_{\nu}(r) \cdot \underset{\mathclap{\text{Filterfunktion}}}{\boxed{S_x(\nu)}}\ d\nu + C \]

\begin{center}
    \begin{tabular}{ccc}
        $S_x(\nu)$      &                       & $m_x$ \\\hline
        $1$             & bolometrisch          & $m_{\mathrm{bol}}$ \\
        Auge            & visuell               & $m_v$ \\
        V-Filter        & V-Filter              & $m_V$ \\
        Fotoplatte      & Fotoplatte            & $m_{\mathrm{pg}}$ \\
        $\vdots$        &                       & $\vdots$ \\\hline
    \end{tabular}
\end{center}

\section{Filtersysteme}
$\rightarrow$ Photometrie

\textsc{Johnson}: U - B - V (ältestes/wichtigstes System)

FWHM: $~100\,\nano\metre$ -- Breitbandfilter

% TODO graphic

\begin{itemize}
    \item Filter schneidet Teil des Spektrums "`aus"'
\end{itemize}

$\Rightarrow$ \textbf{Farben} $m_U =: U; m_V =: V; m_B =: B; \dots$

\fbox{%
    \begin{minipage}{\textwidth}
        \begin{definition}
            Farbindex
            \begin{align*}
                U - B &= m_U - m_B \\
                B - V &= m_B - m_V \\
                \dots
            \end{align*}
        \end{definition}
    \end{minipage}
}

\paragraph{Eichung des Farbsystems} Wega ($T_{\mathrm{eff}} = 10^4\,\kelvin$)
\[ U = B = V = R = \dots = L = 0 \]
\[ \Rightarrow U-B = B-V = V-R = \dots = 0 \]

$\Rightarrow$ Farb-Farbdiagramm (FFD), Farb-Helligkeitsdiagramm (FHD)

% TODO Grafikbeispiele HRD
