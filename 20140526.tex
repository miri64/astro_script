\subsection{Direkte Beobachtung}
\begin{goal}
    "`Foto"' von der 2. Erde in der habitablen Zone (HZ)
\end{goal}

\paragraph{Problem} Stern $\leftrightarrow$ Planeten: Kontrastunterschied
(Stern überstrahlt den Planeten)

% TODO Skizze

\textsc{Wien}'sches Verschiebungsgesetz:
\[ T \cdot \lambda_{\max} = 0.29\,\milli\kelvin = \textbf{const} \]

\begin{itemize}
    \item IR -- bessere Unterscheidung
    \item weite Systeme: Räumliche Seperation Stern -- Planet
\end{itemize}

\paragraph{Problem} Planet im Allgemeinen kühl und leuchtschwach\\
$\hookrightarrow$ junge Planetensysteme (wegen $f_Q$)

\paragraph{Beispiele}
\begin{itemize}
    \item 2M1207\\
        Planetenkanditat um braunen Zwerg
    \item HR8799: Sonnensystem in massereicher "`Ausführung"'
        \begin{align*}
            b &: 10\,M_{\jupiter} \\
            c &: 10\,M_{\jupiter} \\
            d &: 7\,M_{\jupiter} 
        \end{align*}
\end{itemize}

\chapter[Doppelsterne]{Doppelsterne (DS)}
\section{Klassen von DS}
\begin{itemize}
    \item visuelle DS (direkte Beobarchtung)
    \item astronometrische DS
    \item spektroskopische DS
        \begin{itemize}
            \item Single line
            \item double line
        \end{itemize}
    \item Bedeckungsveränderlichkeit (Transit)
\end{itemize}

\section{Roche-Lobe-Massentransfer}
% TODO Skizze

\paragraph{Annahme} $M_{*,1} = M_{*,2}$; Gas: H (1-atomig)
\begin{align*}
    R &: \text{Radius-$RL$} \\
    y &= R - d \\
    A &= \pi x^2
\end{align*}

\paragraph{Annahme} $\boxed{d \ll R}$
\begin{align*}
    x^2 + \left(y + \frac{d}{2}\right)^2 &= R^2 \\
    x^2 + \left(R - d + \frac{d}{2}\right)^2 &= x^2 \left(R - \frac{d}{2}\right)^2 = R^2 \\
    \Rightarrow x^2 + \cancel{R^2} - \left(\cancel{R} - \frac{d}{2}\right)^2 &= R \cdot d - \underset{\approx 0}{\frac{d^2}{4}} \approx R \cdot d \\
    \Rightarrow x &= \sqrt{R \cdot d}
\end{align*}

\paragraph{Massentransfer}
\begin{align*}
    \dot M &= A \cdot \rho \cdot v \\
           &= \pi \cdot R \cdot d \cdot \rho \cdot v
\end{align*}

$v = v_{\mathrm{th}}$:
\begin{align*}
    \frac{m_{\mathrm{H}}}{\cancel{2}} v_{\mathrm{th}}^2 &= \frac{3}{\cancel{2}} kT \\
    v_{\mathrm{th}} &= \sqrt{\frac{3kT}{m_{\mathrm{H}}}} \\
    \Rightarrow \dot M &= \pi \underbrace{\underset{\uparrow}{\rho} R d \sqrt{\frac{3k\overset{\downarrow}{T}}{m_{\mathrm{H}}}}}_{\text{von $d$ abhängig}} \sim \boxed{d} \\
                       &\Rightarrow \boxed{\dot M \sim d^3}
\end{align*}

\part{Astronomische Beobachtungsgrößen}
\chapter{Astronomische Spektroskopie}
\section[Astronomische Helligskeitsysteme]{Astronomische Helligskeitsysteme (Bsp. Sterne)}
\section{Filtersysteme}
