\part{Klassische Astronomie}
\chapter{Koordinatensysteme, Zeit, Sternörter}
\section{Grundlagen der sphärischen Astronomie}
\subsection{Bezugssysteme}
\begin{itemize}
    \item Orientierung: Konstellationen am Himmel\\
        $\rightarrow$ \textbf{Problem:} systematische Bewegung\\
        $\Rightarrow$ \textbf{Bezugssystem mit Fixpunkten}
    \item Symmetrie $\mapsto$ Kuppelkoordinaten \\
        $\hookrightarrow$ Fixierung an Drehimpulsvektoren
    \item Koordinatenursprung im Schwerpunkt
\end{itemize}

\subsection{Äyquatorialsystem}
\begin{goal}
    Positionsbeschreibung im Inertialsystem, Fixierung an Drehimpulsvektoren
    der Erdbewegung
\end{goal}

\begin{description}
    \item[Eigendrehimpuls]  $\vec n_e$
    \item[Bahndrehimpuls]   $\vec n_b$
    \item[Einheitsvektor in Richtung Frühlingspunkt]   $\vec n_F$
\end{description}

\begin{equation}
    (\vec n_e \times \vec n_b) = \vec n_F
\end{equation}

% TODO: abbildung

\begin{itemize}
    \item $\alpha$: Rektaszension [$^\mathrm{h}$, $^\mathrm{m}$, $^\second$]
    \item $\delta$: Deklination [$\degree$, $\arcminute$, $\arcsecond$]
    \item Umrechnung: $360\degree \mathrel{\hat=} 24^\mathrm{h} \Leftrightarrow 15\degree \mathrel{\hat=} 1^\mathrm{h}$
    \item \textbf{Position eines astronomischen Objektes}: $\boxed{(\alpha, \delta)\text{-Angabe}}$
    \item Koordinaten Frühlingspunkt: $(0, 0)$
\end{itemize}
