\section{Bemerkungen zu Störeffekten}
\paragraph{Problem} $N$-Körperproblem nicht allgemein lösbar!

\paragraph{Methode}
\begin{itemize}
    \item Ansatz einenr Störungstheorie
    \item numerische Behandlung des $N$-Körperproblems
    \item \sout{statistische Methoden}
\end{itemize}

\paragraph{Aller einfachster Ansatz}
\textbf{Situation:}
\begin{center}
    \begin{tikzpicture}
        \coordinate (M) at (0, 0);
        \node [above] at (M) {$\mathcal{M}$};
        \coordinate (m) at (2, 0);
        \node [above] at (m) {$m$};
        \coordinate (mS) at (5, 0);
        \node [above] at (mS) {$m_S$};
        \filldraw(M) circle (1pt);
        \filldraw(m) circle (1pt);
        \filldraw(mS) circle (1pt);

        \draw (M) -- (m) node [pos=0.5, above] {$a$};
        \draw (m) -- (mS) node [pos=0.5, above] {$d$};
        \draw [->, >=stealth'] (6,1) node [right] {störende Masse} -- (5.5,0.5);
    \end{tikzpicture}
\end{center}

\paragraph{Anmerkung} kleine Störung

\paragraph{Abschätzung} maximaler Einfluss (Konjunktion / Opposition)

\paragraph{Beschleunigungen}
\[ b_m = \frac{G m_S}{d^2}; \qquad b_{\mathcal{M}} = \frac{G m_S}{(a + d)^2} \]

$\Rightarrow$ Störbeschleunigung: $\Delta b = b_m - b_{\mathcal M} = G m_S (\frac{1}{d^2} - \frac{1}{(d + a)^2})$

$a \ll d$: Reihenentwicklung
\[ \boxed{\text{Taylorentwicklung:} f(x) = f(x_0) + f'(x_0) \cdot (x - x_0) + \dots} \]

\begin{align*}
    \frac{1}{x^2} &\approx \frac{1}{x_0^2} - \frac{2}{x_0^3} (x - x_0) + 0, \text{mit} x = a + d; x_0 = d \\
    \Rightarrow \frac{1}{(a+d)^2} &= \frac{1}{d^2} - \frac{2}{d^3} (a + \cancel{d} - \cancel{d}) + 0 \\
    \Rightarrow a b &= \frac{Gm_S 2 a}{d^3} = \frac{2 G \mathcal{M}}{a^2} \cdot f
\end{align*}

\begin{definition}
    Störfaktor $f$: \[ \boxed{f := \frac{m_S}{\mathcal{M}} \cdot \left(\frac{a}{d}\right)^3} \]
\end{definition}

je größer $f$ $\Rightarrow$ wichtiger Einfluss auf Bahnen
\begin{center}
\begin{tabular}{ll}
    Erde $\xrightarrow{\text{durch}}$ Venus: & $f \approx \dfrac{1}{37000}$ \\
    Erde $\xrightarrow{\text{durch}}$ Jupiter: & $f \approx \dfrac{1}{53000}$ \\
    Erde $\xrightarrow{\text{durch}}$ Saturn: & $f \approx \dfrac{1}{360}$
\end{tabular}
\end{center}

$\rightarrow$ Numerische Simulation: $N$-body simulation

\section{Virialsatz}
\begin{framed}
    \begin{theorem}
        Die Gesamtenergie $E$ eines gravitativ gebundenen Systems $(E < 0)$ im 
        Gleichgewicht ist immer gleich der Hälfte der zeitgemittelten potentiellen
        Energie $\langle E_{\mathrm{pot}} \rangle$ des Systems:
        \[ E = \frac{\langle E_{\mathrm{pot}} \rangle}{2} \Leftrightarrow \langle E_{\mathrm{kin}} \rangle = - \frac{\langle E_{\mathrm{pot}} \rangle}{2}\]
    \end{theorem}
\end{framed}

\paragraph{Betrachte}
\begin{align*}
    Q &\underset{\mathclap{
        \begin{array}{c}
            \uparrow \\
            \text{Summe über alle Teilchen $i$ des Systems}
        \end{array}
        }}{=} \sum\limits_{i} \langle\vec p_i, \vec r_i\rangle = \sum\limits_{i} m_i \frac{d}{dt} \langle\vec r_i, \vec r_i\rangle\\
        \Rightarrow \frac{dQ}{dt} &= \sum\limits_{i} (\langle\dot{\vec{p}}_i, \vec r\rangle + \langle\vec p, \dot{\vec{r}}_i\rangle)
        = \frac{d}{dt} \sum\limits_i m_i \underbrace{\langle\dot{\vec{r}}_i, \vec{r}_i\rangle}_{= ?}
\end{align*}

\color{OliveGreen}
\textbf{Nebenrechnung:}
\begin{align*}
    \frac{d}{dt} \langle\vec{r}_i, \vec{r}_i\rangle &= 2 \langle\vec{r}_i, \dot{\vec{r}}_i\rangle = \frac{d}{dt} (r_i)^2 \\
    \Rightarrow \langle\vec{r}_i, \dot{\vec{r}}_i\rangle &= \frac{1}{2} \frac{d}{dt} r_i^2 \\
\color{black}
\Rightarrow \dot Q &\color{black}=\frac{d}{dt} \sum_i \frac{1}{2} \frac{d}{dt} (m_i r_i^2) = \frac{1}{2} \frac{d^2}{dt^2} \underbrace{\sum\limits_i (m_i r_i^2)}_{FI}
\end{align*}
\color{black}
$I$: Trägheitsmement der Teilchen des Systems
\begin{align*}
    \Rightarrow \dot Q &= \frac{1}{2} \ddot I    \tag*{$| - \sum\limits_i \langle \vec p_i, \dot{\vec{r}}_i\rangle$} \\
    \Rightarrow \frac{1}{2}\ddot I - \sum\limits_i \langle \vec p_i, \dot{\vec{r}}_i\rangle &= \sum\limits_i \langle \dot{\vec{p}}_i, \vec{r}_i\rangle = \underbrace{\sum\limits_i \langle\vec F_i \cdot \vec r_i\rangle}_{\text{\textsc{Clausius}' Virial}} \\
    \Rightarrow \frac{1}{2}\ddot I - \underbrace{\sum\limits_i m_i \langle \dot{\vec{r_i}}, \dot{\vec{r}}_i\rangle}_{= 2 E_{\mathrm{kin}}} &= \sum\limits_i \langle \vec{F}_i, \vec{r}_i\rangle \\
\end{align*}

$F_i$: Summe aller auf Teilchen $i$ wirkenden Kräfte

\paragraph{Betrachte} Gravitativ gebundenes System bestehend aus vielen Teilen

Es gilt:
\begin{align*}
    \sum\limits_i \langle\vec F_i, \vec r_i\rangle &= \sum\limits_i \left\langle\sum\limits_{j \neq i} \vec F_{ij}, \underbrace{\vec r_i}_{\mathclap{\qquad\qquad= \frac{1}{2} (\vec r_i - \vec r_j) + \frac{1}{2} (\vec r_i + \vec r_j)}}\right\rangle\\
                &= \frac{1}{2} \sum\limits_i \left\langle\sum\limits_{j \neq i} \vec F_{ij}, \underbrace{\vec r_i - \vec r_j}_{=: - \vec r_{ij}}\right\rangle + \underbrace{\frac{1}{2} \sum\limits_i \left\langle\sum\limits_{j \neq i} \vec F_{ij}, \vec r_i + \vec r_j\right\rangle}_{= 0; \vec F_{ij} = -\vec F_{ji}} \\
    \vec F_{ij} &= \left(\frac{Gm_im_j}{r_{ij}^2}\right) \cdot \vec r_{ij} \tag{Gravitationskraft $i \Leftrightarrow j$}\\
    \Rightarrow \sum\limits_i \langle\vec F_i, \vec r_i\rangle &= -\frac{1}{2} \sum\limits_i\left(\sum\limits_{j \neq i} \left(\frac{Gm_im_j}{r_{ij}^{\cancel{3}}}\right) \underbrace{\langle\vec r_{ij}, \vec r_{ij}\rangle}_{= \cancel{r_{ij}^2}}\right)\\
        &= -\frac{1}{2} \sum\limits_i \left(\sum\limits_{j \neq i} \frac{G m_i m_j}{r_{ij}}\right) = \frac{1}{2} \sum\limits_i \sum\limits_{j \neq i} E_{\mathrm{pot}_{ij}} = E_{\mathrm{pot}}\\
    \underset{=0}{\cancel{\frac{1}{2} \langle\ddot I\rangle}} - 2 \langle E_{\mathrm{kin}} &= \langle E_{\mathrm{pot}}\rangle
\end{align*}

\paragraph{Zeitmittlung} \[\langle \cdot \rangle = \frac{1}{2} \int\limits_0 \cdot\ dt\]

\color{OliveGreen}
\textbf{Nebenrechnung:}
\begin{itemize}
    \item $\langle \ddot I \rangle = \frac{1}{2} \int\limits_0^{\tau} \ddot I\ dt = \dot I |_0^{\tau} = \underbrace{\dot I(\tau) - \dot I(0)}_{=0}$ (periodische Orbit-Mittlung über eine Periode)
    \item System im Gleichgewicht: $I = \mathbf{const} \Rightarrow \dot I = 0$
\end{itemize}
\color{black}


\begin{align*}
    \Rightarrow \langle E_{\mathrm{pot}}\rangle &= -2 \langle E_{\mathrm{kin}} \rangle \\
    \text{oder}\qquad \langle E\rangle &= \langle E_{\mathrm{pot}} \rangle + \langle E_{\mathrm{kin}} \rangle = 
    \langle E_{\mathrm{pot}} \rangle - \frac{1}{2} \langle E_{\mathrm{pot}} \rangle = \frac{\langle E_{\mathrm{pot}} \rangle}{2}
\end{align*}

\paragraph{Anwendung des Virialsatzes}
\begin{itemize}
    \item ideales Gas
    \item Kugelsternhaufen
    \item Galaxiehaufen
    \item Sternentstehung
    \item Himmelsmechanik
    \item ...
\end{itemize}

\part{Planeten}
\chapter{Physikalische Eigenschaften des Sonnensystems}
\section{Stabilität der Atmosphäre}
\paragraph{Frage} Warum cerliert ein Planet seine Atmosphäre?
\paragraph{Einfache Abschätzung} Energiegleichgewicht
\[ E_{\mathrm{kin}} = E_{\mathrm{grav}} \]
$\Rightarrow$ Entweichgeswindigkeit eines Teilchens mit der Masse $m$ an der 
              Planetenoberfläche
\[ \frac{\cancel{m}}{2} v_{\mathrm{esc}}^2 = \frac{G \cancel{m} m_{\mathrm{Pl}}}{R_{\mathrm{Pl}}} \]

\[ \Rightarrow v_{\mathrm{esc}} = \sqrt{\frac{2 G m_{\mathrm{Pl}}}{R_{\mathrm{Pl}}}} \]

Geschwindigkeitsverteilungsfunktion eines Gases
\[ f(\vec v)\ dv^3: \int f (\vec v)\ dv^3 = 1 \tag{Normierung} \]

z. B. \textbf{\textsc{Maxwell}-Geschwindigkeitsverteilung}
\[ f(\vec v) = \left(\frac{m}{2 \pi T}\right)^{\frac{3}{2}} e^{-\frac{mv^2}{2} \per (k T)} \]

\[ \Rightarrow \frac{m v_{\mathrm{th}}^2}{2} \approx kT \Rightarrow \text{Thermische Geschwindigkeit $v_{\mathrm{th}} = \sqrt{\dfrac{2 k T}{m}}$}\]

\paragraph{Bruchteil der entweichenden Teilchen $\delta$}
\begin{align*}
    \delta &:= \int\limits_{\mathclap{\vec k \cdot \vec v > v_{\mathrm{esc}}}} f(\vec v)\ dv^3 \tag{$\vec k$: Normalvektor der Atmosphäre}
\end{align*}

\paragraph{Vorteil} $v_{\mathrm{esc}}$ klein: $m_{\mathrm{Pl}}$ klein, $R_{\mathrm{Pl}}$ groß \\
    $f(\vec v)$ groß: $T$ groß

\begin{definition}
    Stabilitätsparameter
    \[ \Gamma^2 := \left(\frac{v_{\mathrm{th}}}{v_{\mathrm{esc}}}\right)^2 = \left(\frac{kTR_{\mathrm{Pl}}}{G m_{\mathrm{Pl}} m}\right) \]
    \begin{tabular}{r@{:\ }l}
        $\Gamma \gg 1$ & instabil \\
        $\Gamma \ll 1$ & stabil
    \end{tabular}
\end{definition}

\paragraph{Planetenatmosphäre} (Thermischer Verlust) \\
$\Gamma$-Reihenfolge: Jupiter $>$ Saturn $>$ Neptun $>$ Uranus $>$ Erde $>$
Venus $>$ Mars $>$ Pluto $>$ Triton $>$ \emph{Titan} $>$ Io $>$ ... $>$ \emph{Merkur}
$>$ \emph{Mond} $>$ ...

\section{Energiehaushalt und Atmosphärentemperatur}
\paragraph{Anmerkung} Strahlungsgleichgewicht, Schwarzkörper
\[ \boxed{\text{Abstrahlung} = \underset{\begin{matrix}\Uparrow\\\text{Sonne}\end{matrix}}{\text{Einstrahlung}}} \]
 
\subsubsection{Energieabstrahlung der Sonne}
\[ L_{\astrosun} = \underbrace{4 \pi R_{\astrosun}^2}_{\text{Oberfläche}} \cdot \underset{\begin{matrix}\text{Strahlungsfluss}\\\text{[Energie / Zeit / Fläche]}\end{matrix}}{F_{\astrosun} (R_{\astrosun})} \leftarrow \text{Leuchtkraft der Sonne [Energie / Zeit]} \]

\begin{center}
    \begin{tikzpicture}[tdplot_main_coords]
        \tdplotdrawarc[dashed]{(0,0,0)}{2}{90}{270}{}{};
        \draw [->] (0,0,0) -- (0,2,0) node [above, pos=0.5] {$R_{\astrosun}$};
        \shade[ball color=red!10!yellow, opacity=0.50] (0,0,0) circle (2cm);
        \draw (0,0,0) circle (2cm);

        \tdplotdrawarc{(0,0,0)}{2}{-90}{90}{}{};
    \end{tikzpicture}
\end{center}

Schwarzkörper:
\[ B_{\nu}(T) \xrightarrow{\int\ d\nu} \boxed{F_{\astrosun} = \sigma T^4} \tag{Stefan-Boltzmann-Gesetz} \]
$\sigma$: Stefan-Boltzmann-Gesetz
\[ E_{\mathrm{ph}} = h\nu \]

\begin{center}
    \begin{tikzpicture}
        \draw (0,0) -- (7.4,0);

        \draw [domain=-90:90] plot ({cos(\x)}, {sin(\x)});
        \draw [domain=-41.81:41.81] plot ({3*cos(\x)}, {3*sin(\x)});
        \draw [domain=-16.6:16.6] plot ({7*cos(\x)}, {7*sin(\x)});
        \node (r) at (7,0) [anchor=south west] {$r$};
        \draw [->] (0,0) -- (41:1) node [pos=0.5] {$R_{\astrosun}$};
    \end{tikzpicture}
\end{center}

\paragraph{Energieerhaltung}
\begin{align*}
    L_{\astrosun} &= \cancel{4 \pi} R_{\astrosun}^2 F_{\astrosun} = \cancel{4 \pi} r^2 \boxed{f_{\astrosun}(r)}\\
f_{\astrosun}(r) &= F_{\astrosun} (R_{\astrosun}) \cdot \underbrace{\left(\frac{R_{\astrosun}}{r}\right)^2}_{\mathclap{\text{geometrische Verdünnung}}} \tag{Am Ort $r$ ankommender Strahlungsfluss}\\
    f_{\astrosun}(1\,\mathrm{AU}) &= 1.37\,\kilo\watt\metre^{-2} \tag{Solarkonstante}
\end{align*}

\paragraph{Absorbierte Energie}
\begin{align*}
L_{\mathrm{abs}} &= f_{\astrosun}(r) \cdot \underset{\mathclap{\begin{matrix}\uparrow\\\text{absorbierende Fläche}\end{matrix}}}{\mathcal{F}_{\mathrm{abs}}} \cdot (1 - A) + \underbrace{4 \pi R_{\mathrm{Pl}} f_Q}_{\mathclap{\text{innere Energiequelle}}}\\
    \text{Albedo $A$:}\ A &= \frac{\text{gestreutes / reflektiertes Licht}}{\text{einfallendes Licht}}
\end{align*}

\paragraph{Emittierte Energie}
\begin{align*}
L_{\mathrm{em}} &= f_{\mathrm{em}} \cdot \mathcal{F}_{\mathrm{em}} = \underbrace{\sigma \cdot T_{\mathrm{Pl}}^4}_{\mathclap{\begin{matrix}\downarrow\\\text{Abstrahlung erfolgt in $\mathbb{R}$}\end{matrix}}}\cdot \mathcal{F}_{\mathrm{em}} \cdot e^{- \tau_{\mathrm{IR}}}
\end{align*}
$\tau_{\mathrm{IR}}$: optische Tiefe

Gegeben: $L_{\mathrm{em}} = L_{\mathrm{abs}}$
\[ \cancel{\sigma} \cdot \left(\frac{R_{\astrosun}}{r}\right)^2 \cdot T_{\astrosun}^4 \cdot \mathcal{F}_{\mathrm{abs}} \cdot (1-A) + 4 \pi R_{\mathrm{Pl}} f_Q = \cancel{\sigma} \cdot T_{\mathrm{Pl}}^4 \cdot \mathcal{F}_{\mathrm{E}} \cdot e^{\tau_{IR}} \]

Gesucht: Strahlungsgleichgewichtstemperatur des Planeten ($f_Q = 0, \tau_{IR} = 0$)
\[ \boxed{T_{\mathrm{Pl}^2} = (1 - A) \left(\frac{\mathcal{F}_{\mathrm{abs}}}{\mathcal{F}_{\mathrm{em}}}\right) \left(\frac{R_{\astrosun}}{r}\right)^2 T_{\astrosun}^4} \]

Annahme:
\begin{align*}
    \mathcal F_{\mathrm{abs}} &= \pi R_{\mathrm{Pl}}^2 \\
    \mathcal F_{\mathrm{em}} &= \begin{cases}
        4 \pi R_{\mathrm{Pl}}^2, & \text{schnelle Rotation} \\
        2 \pi R_{\mathrm{Pl}}^2, & \text{langsame Rotation}
    \end{cases}
\end{align*}

Beispiel: Erde $A \approx 0.3$, $f_Q \approx 0.06 \watt\metre^{-2} \approx 10^{-4} f_{\astrosun}$
\begin{align*}
    r &= 1\,\mathrm{AU} & R_{\astrosun} &= 696000\,\kilo\metre & \\
    L_{\astrosun} &= 3.85 \cdot 10^{26}\,\watt & T_{\astrosun} &= 5780\,\kelvin
\end{align*}
\[ \Rightarrow T = \sqrt{3.787 \cdot 10^{-6}}\,T_{\astrosun} \approx 0.044\,T_{\astrosun} \approx 255\,\kelvin \]

\paragraph{Folgerung} tatsächliche Planetentemperatur $T_{\mathrm{Pl}}$ größer
als Strahlungsgleichgewichtstemperatur $T$
\[ \Rightarrow \boxed{T_{\mathrm{Pl}} = T} \]
