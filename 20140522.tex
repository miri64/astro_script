\subsection{Transitmethode}
\paragraph{Prinzip} Bedeckung des Sterns des Planeten\\
$\Rightarrow$ "`Einbruch"' in der Lichtkurve

\paragraph{Problem} Auswahleffekt $i$ im Bereich von $90\degree$
\paragraph{Vorteil} sehr kleine Planeten beobachtbar
\paragraph{Beispiel} HD209458 b
\begin{align*}
    i &= 87.1\degree & R_{\Pl} &= 1.27\,R_{\jupiter}
\end{align*}
\paragraph{RV-Messung} $M_{\Pl} \cdot \sin i = 0.63\, M_{\jupiter}$
\begin{align*}
    \Rightarrow M_\Pl &= \frac{0.63}{0.9987}\,M_{\jupiter} & M_{\jupiter} &= 1.197 \cdot 10^{30}\,\gram \\
    R_\Pl &= 1.27\,R_{\jupiter} = 9.067 \cdot 10^{7}\,\metre \\
    \Rightarrow \rho_\Pl &= \frac{M_\Pl}{\frac{4}{3} \pi R_\Pl^3} \approx 0.38\,\frac{\gram}{\metre^3} \tag*{$\rightarrow$ weniger dicht als Saturn}
\end{align*}

\paragraph{Satelliten-Missionen}
\begin{itemize}
    \item COROT
    \item Kepler
    \item PLATO (2024)
\end{itemize}

\subsection{Microlensing Methode}
